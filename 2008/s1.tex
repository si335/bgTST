\documentclass[../TST.tex]{subfiles}
\begin{document}
\begin{large}
	\textbf{Short Exam 1}
\end{large}
\begin{sproblem}
A solid ball of mass $m$ and radius $a$ (moment of inertia $I=\frac{2}{5}ma^2$) starts rolling without slipping from the top of another fixed ball of radius $b$. Its initial velocity is negligible. The acceleration due to gravity is $g$.
\end{sproblem}

\begin{subpart}
	\item Find the angle $\theta=\theta_0$ at which the rolling ball will lose contact with the fixed ball. The angle $\theta$ is measured between the upward direction and the segment connecting the centres of the balls.
	\item Find the velocity  of the centre of mass $v$ of the rolling ball when it detaches.
	\item What coefficient of friction $k$ would make the upper ball start slipping at an angle $\theta=\alpha<\theta_0$?
\end{subpart}

\ifprob \else
\begin{solution}
 (a) Let us find the dependence of the centre of mass velocity $v$ on $\theta$. Our ball rolls without slipping, so the friction force at the point of contact with the other ball does not do any work. Applying conservation of energy,
\begin{equation*}
	\frac{mv^2}{2} + \frac{I\omega^2}{2} = mg(a+b)(1-\cos{\theta}),
\end{equation*}
and using $v=\omega a$ (no slipping), we find that $v^2=\frac{10}{7}g(a+b)(1-\cos{\theta})$. Now we can determine the normal force $N(\theta)$. There are three forces on the ball: gravity $mg$, friction $f$, and a normal force $N$. Projecting these along the normal, we find 
\begin{equation*}
	mg\cos{\theta}-N=ma_n,
\end{equation*}
where $a_n=v^2/(a+b)$ is the normal acceleration of the centre of mass. After substituting for $v^2$, we get $N=\left(\frac{17}{7}\cos{\theta}- \frac{10}{7}\right)mg$. The ball loses contact with the surface when $N=0$. Thus $\cos{\theta_0}=\frac{10}{17}$, or $\boxed{\theta_0\approx \ang{54}.}$\\

(b) Plugging the value for $\theta$ back into the formula for $v(\theta)$, we get \fbox{$v=\sqrt{\frac{10}{17}g(a+b)}.$}\\

(c) First, we need to find the friction force $f(\theta)$. Projecting forces along the tangent, we obtain
\begin{equation*}
	mg\sin{\theta}-f=ma_\tau
,
\end{equation*}
where the tangential acceleration $a_\tau$ of the centre of mass is related to the angular acceleration of the ball $\varepsilon$ by $a_\tau=\varepsilon a$ (no slipping). Taking torques about the centre of mass, we get $fa=I\varepsilon$. Now, solving for $f$, we find $f(\theta)=\frac{2}{7}mg\sin{\theta}$.  For the ball to start slipping at $\theta=\alpha$, we need $f(\alpha)=kN(\alpha)$. Plugging in our formulae for $f$ and $N$, we get 
\begin{equation*}
	\boxed{k=\frac{2\sin{\alpha}}{17\cos{\alpha}-10}}\,.
\end{equation*}
You would be well advised to verify that all the results behave appropriately in special cases.
\end{solution}
\fi
\ifprob
	\clearpage
	% \vspace*{1cm}
\else
	\clearpage
\fi
\end{document}

