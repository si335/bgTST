\documentclass[../TST.tex]{subfiles}
\begin{document}
\begin{pproblem}[Charged disc in a magnetic field]
A uniform dielectric disc of mass $m$ and charge $q$ is initially at rest. The disc is placed in a magnetic field parallel to its axis. The field is arbitrary, $B=B(t)$, with $B(0)=0$. Find the time dependence of the angular velocity of the disc $\omega(t)$.
\end{pproblem}

\ifprob \else
\begin{solution} Denote the radius of the disc by $R$ and its charge density by $\sigma=\frac{q}{\pi R^2}$. In keeping with the symmetry of the problem, consider a circular loop of radius $r<R$. This loop is pierced by a magnetic flux $\Phi=B\pi r^2$. The magnetic field is always changing, so by Faraday's law we have an electric field $E$ along the loop:
	\begin{equation*}
	E(r)\cdot 2\pi r = - \frac{\mathrm{d}\Phi}{\mathrm{d}t} = -\frac{\mathrm{d}B}{\mathrm{d}t}\pi r^2
	.
	\end{equation*}
A surface element at $r$ which carries a charge $\mathrm{d}q$ will then feel a force $E\,\mathrm{d}q$, which creates a torque $E r\,\mathrm{d}q$ about the rotation axis. The net torque from the ring between $r$ and $r+\mathrm{d}r$ is 
\begin{equation*}
\mathrm{d}M=Er( 2\pi r\sigma\,\mathrm{d}r)= -\frac{\mathrm{d}B}{\mathrm{d}t}\left(\frac{\pi r^2}{2\pi r}\right)r\,(2\pi r\sigma\,\mathrm{d}r).
\end{equation*}
The total torque from the whole disc is
\begin{equation*}
M = \int \mathrm{d}M = -\frac{\mathrm{d}B}{\mathrm{d}t}\sigma \left(\frac{\pi R^4}{4}\right) = -\frac{\mathrm{d}B}{\mathrm{d}t}\left(\frac{qR^2}{4}\right) 
.
\end{equation*}
This is equal to $I \frac{\mathrm{d}\omega}{\mathrm{d}t}$, where $I=\frac{1}{2}mR^2$ is the moment of inertia of the disc. The radius cancels out, and we're left with
\begin{equation*}
\mathrm{d}\omega = -\mathrm{d}B \frac{q}{2m}
.
\end{equation*}
The initial angular velocity and the initial magnetic field are both zero, so the final answer is 
\begin{equation*}
\boxed{\omega(t) = -\frac{qB(t)}{2m} .}
\end{equation*}
The minus sign implies that the angular velocity is directed opposite to the magnetic field.
\end{solution}
\fi

\end{document}
