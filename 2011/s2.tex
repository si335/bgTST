\documentclass[../TST.tex]{subfiles}
\begin{document}
\begin{large}
	\textbf{Short Exam 2}
\end{large}

\begin{sproblem}
A circuit consists of a battery of EMF $E$, a capacitor $C$, a resistor $R$, and an open switch, all in series. At time $t=0$ the switch is closed.
\begin{subpart}
	\item Find the magnitude $I_1(0)$ of the initial current, the time dependence of the current $I_1(t)$, and the total heat $Q_1$ dissipated at the resistor until the capacitor is fully charged. \hphantom{..............} \hfill \score{2.0}
\end{subpart}
The capacitor is discharged and the resistor is replaced by a nonlinear element with an I-V curve $I=\beta U^{3/2}$, where $\beta$ is a constant. At time $t=0$ the switch is closed.
\begin{subpart}[resume]
	\item Find the magnitude $I_2(0)$ of the initial current, the time dependence of the current $I_2(t)$, and the total heat $Q_2$ dissipated at the nonlinear element until the capacitor is fully charged. \score{3.0}
\end{subpart}
\end{sproblem}

\ifprob \else
\begin{solution} 
(a) Initially the charge $q$ on the capacitor is zero, likewise for its voltage. The voltage loop rule for the circuit is
\begin{equation*}
E-\frac{q}{C}-IR=0
.
\end{equation*}
Initially we just have $E-I_1(0)R=0$, or \fbox{$I_1(0)=\frac{E}{R}$.} In general, $I=\frac{\mathrm{d}q}{\mathrm{d}t}$, so
\begin{equation*}
\frac{\mathrm{d}q}{\mathrm{d}t}+\left( \frac{1}{RC}\right)q=\frac{E}{R}
.
\end{equation*}
The general solution is $q(t)=Ae^{-\frac{t}{RC}}+CE$, where $A$ is an arbitrary constant. Using the initial condition $q(0)=0$, we obtain
\begin{equation*}
	q(t)=CE\left(1-e^{-\frac{t}{RC}}\right) \quad\Rightarrow\quad \boxed{I(t)=\left(\frac{E}{R}\right) e^{-\frac{t}{RC}}.}
\end{equation*}
We can now find the dissipated heat $Q_1$ by integrating $\int_0^\infty I^2R\,\mathrm{d}t$, but the algebra can be avoided. The point is that $Q_1$ is equal to the total energy input from the battery, minus the energy stored in the capacitor in the final state (when its charge is $q_0=CE$). Then,
\begin{equation*}
	Q_1=q_0E-\frac{q_0^2}{2C}=\boxed{\frac{CE^2}{2}.}
\end{equation*}
(b) Let the voltage on the nonlinear element be $U_N$, so that the current in the circuit is $I=\beta U_N^{3/2}$. As in the previous part, initially $q/C=0$, so $U_N=E$, and therefore  \fbox{$I_2(0)=\beta E^{3/2}$.} The general voltage loop rule is
\begin{equation*}
E-\frac{q}{C}-U_N=0
.
\end{equation*}
We want to involve the current $I$, so we differentiate this equation to get
\begin{equation*}
-\frac{I}{C}-\frac{\mathrm{d}U_N}{\mathrm{d}t}=0
.
\end{equation*}
Now $U_N=(I/\beta)^{2/3}$, hence
\begin{equation*}
	\frac{\mathrm{d}I}{\mathrm{d}t}=-\left(\frac{3\beta^{2/3}}{2C}\right) I^{4/3}
.
\end{equation*}
After integrating, we are left with 
\begin{equation*}
	I(t)=\left(B+\frac{\beta^{2/3}}{2C}t\right)^{-3}
,
\end{equation*}
where $B$ is a constant. After imposing the initial condition for the current,
\begin{equation*}
	\boxed{I(t)=\beta \left(\frac{1}{\sqrt{E}}+ \frac{\beta t}{2C}\right)^{-3}.}
\end{equation*}
The dissipated heat can now be found by integrating $\int_0^\infty U_NI\,\mathrm{d}t$, but we can save ourselves the work by using the same trick as before. The answer is, again, \fbox{$Q_2=\frac{CE^2}{2}$.}
\end{solution}

\fi
\ifprob 
	\clearpage
\else 
	\vspace*{10mm}
	% \clearpage
\fi
\end{document}

