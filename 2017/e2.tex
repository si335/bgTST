\documentclass[../TST.tex]{subfiles}
\begin{document}
\begin{eproblem}[Conductivity of liquids]{\ \\[5pt]}
\textit{Equipment:}\\
Alternating voltage source, 2 multimeters, wires, paperclips of unknown radius $r$, tub, ruler, 3 bottles with $\qty{1.5}{l}$ of tap water in each, kitchen scale, table salt (NaCl), tape, scissors, aluminium foil (be careful with the sharp edges), funnel, graph paper. \\

All inductances and capacitances in the circuits below are negligible. Record all measurements in tables. Write down your results in the answer sheet.\\

\textit{Task 1. Conductivity of tap water.}\\
Consider two identical infinite cylindrical conductors of radius $r$. The distance between the conductors is $L$. If the conductors pierce an infinite weakly conducting layer normally, some resistance $R$ will be measured between them. If the thickness of the layer is $h$ and its conductivity is $\sigma$, the resistance is given by
\begin{equation}
	R=\frac{C}{\sigma h}\ln{\left(\frac{L}{r}\right)}
,
\end{equation}
where $C$ is an unknown constant. \\

Pour the three bottles of water into the tub. Assemble a circuit that will allow you to measure the resistance between two vertical paperclips in a layer of water.  The multimeters should be connected in a way that allows you to measure small resistances. Use AC voltage of RMS value $U=\qty{4}{V}$ and frequency $\nu=\qty{2}{kHz}$.
\begin{enumerate}
	\item  Measure the resistance between the paperclips $R$ at different distances $L\in [\qty{1}{cm}, \qty{5}{cm}]$ in steps of $\qty{0.5}{cm}$. Keep the paperclips as far away as possible from the sides of the tub. Present your results in a table.\score{4.0}
	\item  Stick a sheet of foil to each of the two smaller sides of the tub. Measure the resistance of the tap water between the sheets $R$. Find the conductivity of the tap water $\sigma$. Record the necessary length measurements in the answer sheet as well.  \score{2.0}
	\item  Using Equation (1) and your value for the conductivity of tap water $\sigma$ from Task 1(b), plot your data from Task 1(a) in appropriate variables and find the radius of the paperclips $r$, as well as the constant $C$. \score{3.0}\\
\end{enumerate}

\textit{Task 2. Conductivity of salt salutions.}\\
Add salt to the water and study four solutions of mass fractions $c=\qty{0.50}{\percent}$, $\qty{1.00}{\percent}$, $\qty{1.50}{\percent}$, $\qty{2.00}{\percent}$ one after the other. 

\begin{enumerate}
	\item Use the sheets of foil from Task 1(b) to measure the conductivity $\sigma$ of these solutions. Assume that it is given by
	\begin{equation}
	\sigma = \alpha c
	.
	\end{equation}
	Plot a graph of the conductivity $\sigma$ against the mass fraction $c$ of the salt solutions. Using the graph, find the parameter $\alpha$. \score{4.0}
	\item Equation (2) does not agree well with the experimental data. Instead assume a more general dependence of the form
\begin{equation}
\sigma = \beta c^n
.
\end{equation}
Plot another graph in the appropriate variables and use it to find the number $n$.  \score{2.0} 
\end{enumerate}

Call the examiner if you suspect that a multimeter's fuse has blown, or in case of any other technical difficulties.
\end{eproblem}
\end{document}
