\documentclass[../TST.tex]{subfiles}
\begin{document}
\begin{large}
	\textbf{Short Exam 2}
\end{large}

\begin{sproblem}
A point charge $q$ of mass $m$ is initially at rest. The charge is placed in a homogeneous electric field $E$ and a homogeneous magnetic field $B$ at right angles to $E$. It turns out that the charge's trajectory is a periodic curve, i.e. the trajectory will consist of many identical segments. Note that the charge's motion can be represented as a superposition of two simpler sorts of motion.
\begin{subpart}
	\item Find the time $T$ in which the charge covers one segment of its trajectory. In other words, find the time interval between two consecutive instants at which the charge is at rest. 
	\item Find the spatial periodicity of the curve $L$. In other words, find the distance between two neighbouring positions where the charge is at rest.
	\item Find the maximum velocity of the charge $v_\mathrm{max}$.
	\item Find the `width' of the trajectory $H$.
	\item What are the two simpler sorts of motion that give rise to the resultant motion of the charge? What are their parameters?
	\item Find the radius of curvature $r_\mathrm{curv}$ of the trajectory at the points of maximum velocity $v_\mathrm{max}$.
\end{subpart}
\end{sproblem}

\ifprob \else
\begin{solution} 
(a) We will assume that the electric field $\mathbf{E}$ is directed along $\hat{\mathbf{x}}$  and that the magnetic field $\mathbf{B}$ is directed along $\hat{\mathbf{z}}$. The total force on the charge is $\mathbf{F}=q\mathbf{E}+q\mathbf{v}\times\mathbf{B}$. This force is perpendicular to $B$, so the motion is constrained to the $xy$-plane. We want to find equations for the velocity components $v_x$ and $v_y$. Writing the fields as $\mathbf{E}=E \hat{\mathbf{x}}$ and $\mathbf{B}=B \hat{\mathbf{z}}$, we find
\begin{align}
	ma_x&= qv_yB+qE, \label{fir}\\
	ma_y&=-qv_xB. \label{sec} 
\end{align}
We differentiate \eqref{fir} and plug in \eqref{sec}, resulting in
\begin{equation*}
	\ddot{v}_x=-\left(\frac{qB}{m}\right)^2v_x
.
\end{equation*}
Setting $\frac{qB}{m}\equiv\omega_c$ and using the initial condition $v_x(0)=0$, we get the solution
\begin{equation*}
	v_x=v_{0x}\sin{(\omega_ct)}
.
\end{equation*}
To obtain $v_{0x}$, we substitute our solution into \eqref{fir} and apply $v_y(0)=0$. Then $v_{0x}=\frac{E}{B}$. We end up with the following equations for the velocity:
\begin{align*}
	v_x&= \frac{E}{B}\sin{(\omega_c t)},\\
	v_y&= \frac{E}{B}\left(\cos{(\omega_ct)}-1\right). 
\end{align*}
The charge is at rest $t=0$, and it's at rest again when $t$ is such that both $\sin{(\omega_c t)}$ and $\cos{(\omega_c t)}$ take the same values. This happens at 
\begin{equation*}
T=\frac{2\pi}{\omega_c}=\boxed{\frac{2\pi m}{qB}.}
\end{equation*}
(b) After integrating the equations for the velocity and applying the initial conditions $x(0)=0$ and $y(0)$ to fix the integration constants, we get
\begin{align*}
	x&= \frac{E}{B\omega_c}\left(1-\cos{(\omega_c t)}\right) ,\\
	y&= \frac{E}{B\omega_c}\sin{(\omega_ct)}-\frac{E}{B}t. 
\end{align*}
Using $t=T$, we find that $x$ has stayed the same as at $t=0$, while $y=-\frac{E}{B}T$. The distance $L$ we seek is the absolute value of this shift, so
\begin{equation*}
	\boxed{L=\frac{2\pi mE}{qB^2}.}
\end{equation*}
(c) The square of the velocity at time $t$ is 
\begin{equation*}
	v^2=v_x^2+v_y^2=2\left(\frac{E}{B}\right)^2(1-\cos{(\omega_ct)})
.
\end{equation*}
This is maximised when $\cos{(\omega_ct)}=-1$, meaning $\omega_c t = \pi+2k\pi$ for integer $k$. We conclude
\begin{equation*}
\boxed{v_\mathrm{max}=\frac{2E}{B}.}
\end{equation*}
(d) We see that $y$ will drift to $-\infty$, but $x$ is always confined between $0$ and $\frac{2E}{B\omega_c}$. The trajectory's width is therefore
\begin{equation*}
	\boxed{H=\frac{2mE}{qB^2}.}
\end{equation*}
(e) Along $y$ there's \fbox{uniform motion} with velocity $\mathbf{v_d}=-\frac{E}{B} \hat{\mathbf{y}}$. Additionally, there's \fbox{rotation} with an angular frequency $\omega_c=\frac{qB}{m}$ along a circle of radius $\frac{E}{B\omega_c}=\frac{mE}{qB^2}$. The centre of rotation is a point with coordinates ($\frac{mE}{qB^2}$, $-\frac{E}{B}t$).\\

(f) Referring to the equations for the velocity, we note that at the instant of maximum velocity we have $v_x=0$ and $v_y=-\frac{2E}{B}$. The equations of motion then imply $F_x=-qE$, $F_y=0$. The radius of curvature can now be found from $|F_x|=\frac{mv_\mathrm{max}^2}{r_\mathrm{curv}}$. Thus
\begin{equation*}
	\boxed{r_\mathrm{curv}=\frac{4mE}{qB^2}.}
\end{equation*}


\end{solution}
\fi
\ifprob 
	\clearpage
\else 
	\vspace*{5mm}
\fi
\end{document}

